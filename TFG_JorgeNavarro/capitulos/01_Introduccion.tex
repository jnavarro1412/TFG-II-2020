\chapter{Introducción}
Desde el inicio de nuestra vida, el cerebro va adquiriendo pequeños conocimientos y forma para nosotros las nociones que necesitamos para sobrevivir en el mundo que nos rodea. A medida que vamos creciendo, el sistema educativo hace que en cada curso tengamos que cumplir unos objetivos formativos, que aumentan progresivamente su dificultad. En cada uno de nosotros se plantean una serie de retos y dificultades propias que pueden darse en mayor o menor medida, y ser semajantes a las que sufren personas de la misma condición. 
\\

El proyecto que se plantea para este trabajo parte de un propósito principal, desarrollar un sistema para la ayuda en la enseñanza de personas que se ven afectadas por trastornos del aprendizaje, más concretamente, la dislexia.
\\

Durante la realización del proyecto, se han aplicado la base de algunos conocimientos estudiados en los años Universidad, siendo necesario la adquisición de nuevos conceptos y estudio de tecnologías diferentes a las vistas anteriormente, en especial, en lo referente a la rama de diseño electrónico y programación.
\\

Una de las ideas principales de este sistema es poder ser facilmente utilizable con cualquier persona que tenga síntomas similares, sin exigir una alta capacidad físicas o motoras, y poder ser aplicado para ayudar en la vida diaria de las personas. También es importante que el sistema sea portátil y económico, teniendo en cuenta que no todo el mundo tiene acceso a medicinas o tratamientos especiales, ni tampoco a tecnologías avanzadas que puedan servir de apoyo.

\section{Motivación}
Durante los años previos a la realización de este proyecto, he tenido la oportunidad de trabajar en la formación deportiva de muchas personas, en un rango de edad muy amplio que, en algunos casos, mostraban dificultades en el momento de procesar la información proporcionada, para adquirir conocimientos, o para realizar movimientos fisicos coordinados. En muchas ocasiones, esto causa un sentimiento de impotencia y frustación en quienes desean participar en una actividad y su cerebro crea un desorden que no les permite realizar acciones con soltura y rapidez. 
\\

Esto también sucede en la vida diaría de muchas personas que se ven afectadas en tareas como leer una noticia, orientarse en un mapa o conducir un automóvil siguiendo las indicaciones de un navegador. Por ello, poder realizar una unión entre la solución a este problema y mis estudios para poder ayudar a la evolución de quién pacede este tipo de trastornos, ofrece una oportunidad desafiante. 

\section{Objetivos}
Los objetivos de este proyecto reunen diversos conocimientos y se dirigen al desarrollo del sistema final, con el fin de tener una herramienta que permita ayudar y mejorar la vida cotidiana de las personas. De esta forma se pretende que: 
Los motivos por los que he elegido Nuxeo frente a Alfresco son los siguientes:

\begin{itemize}
	\item Aprovechar la libertad que Arduino ofrece como herramienta de desarrollo e interación, así como su facilidad de adaptación como sistema empotrado.
	\item La tecnología que impulsa Arduino es de distribución de hardware y software libre, siendo posible su estudio, mejoras y distribución para cualquier fin aplicable a trastornos similares.
	\item Diseño del esquema electrónico y creación del prototipo.
	\item Desarrollo de algoritmos que permitan el funcionamiento del sistema.
	\item Anticipar la detección de trastornos del aprendizaje en temprana edad, como también apoyo y ayuda en las etapas más avanzadas.
	\item Ofrecer la oportunidad de mejorar la calidad la vida de los posibles usuarios futuros del sistema, mediante una serie de actividades que permitan corregir los problemas más comunes asociados a la dislexia.
\end{itemize}

\section{Estructura del proyecto}
Este proyecto se estructura en los capítulos que se exponen a continuación, junto a un breve resumen del contenido en cada uno de ellos.
\\

Este primer capítulo, \textbf{Introducción}, se ha realizado una presentación del proyecto. Incluye una explicación donde se exponen las motivaciones que han dado lugar a realizar el mismo. También se muestran los objetivos perseguidos y los pasos a realizar para conseguirlos. Por último, se hace una organización del proyecto en los diferentes capítulos. 
\\

El segundo capítulo, \textbf{Estado del arte}, realiza una introducción a los trastornos del aprendizaje que permite comprender el contexto de aplicación del proyecto. Por otra parte, se encuentra una descripción sobre técnicas y sistemas actuales existentes en la ayuda a la dislexia, así como la propuesta que se expone en el presente proyecto.
\\

El tercer capítulo, \textbf{Materiales y métodos}, comienza con una descripción de los componentes del módulo hardware y las herramientas que permiten su diseño y simulación. En la segunda parte, se muestra el software utilizado para la realización del proyecto.
\\

El cuarto capítulo, \textbf{Desarrollo}, muestra la descripción del proceso de diseño y desarrollo del proyecto en las partes hardware y software.
\\

El quinto capítulo, \textbf{Resultados}, ofrece el resultado final obtenido, con imágenes del prototipo construido. 
\\

El sexto capítulo, \textbf{Planificación y recursos}, muestra la organización llevada a cabo para el desarrollo completo del proyecto, sus etapas y se muestran los costes de los recursos utilizados.
\\

El séptimo capítulo, \textbf{Conclusiones}, emite una valoración final del proyecto y las ideas que hemos adquirido durante la realización.
\\

El octavo capítulo, \textbf{Ampliaciones y mejoras}, ofrece una serie de posibles actualizaciones que se podrían realizar en este proyecto para mejorar su calidad, ampliar su eficacia o ser más competitivo dentro del mercado.