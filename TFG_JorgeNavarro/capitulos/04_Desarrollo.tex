\chapter{Desarrollo del proyecto}
El presente capítulo expone todo lo relacionado con el desarrollo de este proyecto, las herramientas utilizadas para su organización y planificación, la gestión del software con su correspondiente control de versiones y la parte software.
\\

Para la parte hardware se muestran los componentes que constituyen el dispositivo prototipo, y razonando el uso de cada uno de ellos en las diferentes partes. También el diseño, simulación, y pruebas para el correcto funcionamiento del mismo.
\\

En la parte de software se explicará la estructura del programa desarrollado, explicando los algoritmos e implementación que permiten completar el diseño final.

\section{Organización}
En el desarrollo de un proyecto, la organización del mismo es un punto de vital importancia si se quiere completar con éxito. De esta forma, podemos fijar los objetivos que se pretenden cumplir y los niveles de prioridad de cada uno. También permiten localizar defectos que suceden durante el periodo de pruebas
\\

La gestión de las diferentes tareas a cumplir durante el desarrollo de un proyecto requiere de herramientas que nos permitan ver si los objetivos se cumplen. En este proyecto, se ha utilizado la plataforma Trello, que permite crear tableros de trabajo donde se añaden tareas, que se pueden mover entre listas que determinan si una tarea ya está realizada, se está realizando o aún está pendiente de realizar. De esta forma se facilita la organización del proyecto, tanto de forma individual como si se trabaja en equipo.
\\

La escritura de un informe final dónde se detallan los aspectos más relvantes del proyecto, es un punto vital para recoger la información e ideas desarrolladas. Esta memoria ha sido redactada mediante el sistema de composición de textos Latex con la plataforma Overleaf, que ofrece las herramientas para gestionar y ver el estado del documento de forma online, instantánea y gratuita. 
\\
\begin{figure}[h]
  \centering
  \includegraphics[width=0.3\textwidth]{imagenes/overleaf.png}\\[1cm]
\end{figure}

\section{Control de versiones}
Para el desarrollo del software, es fundamental utilizar herramientas que permitan anotar cambios y llevar un control de desarrollo para poder realizar un diseño incremental. En este proyecto se ha utilizado Git, aprovechando el almacenamiento online que ofrece la plataforma GitHub. En el enlace \cite{GitProyecto} se encuentra la dirección asociada al repositorio de este proyecto.
\\

\begin{figure}[h]
  \centering
  \includegraphics[width=0.3\textwidth]{imagenes/github.png}\\[1cm]
\end{figure}

\section{Hardware}
\subsection{Diseño y esquemático}
\subsubsection{Arduino general}
\subsubsection{Conexión con pantalla LCD}
\subsubsection{Conexión con teclado}
\subsection{Simulación}
\subsection{Montaje}

\section{Software}
