\chapter{Estado del arte: ¿Qué hay ya implementado?}
En este capítulo trataremos de comprender los trastornos relacionados con el aprendizaje, entre ellos la dislexia, analizando sus causas, consecuencias y tipologías. También se realizará un estudio de los sistemas actuales que permiten ayudar en el tratamiento de las personas que lo sufren.

\section{Dislexia: tipos, causas y consecuencias}
La Organización Mundial de la Salud define la dislexia como un trastorno específico de la lectura cuyo rasgo principal es una dificultad específica y significativa en el desarrollo de las habilidades para la lectura que no puede explicarse únicamente por la edad mental, problemas de precisión visual o una escolarización inadecuada. Varias capacidades se ven afectadas: comprensión lectora, reconocimiento de palabras escrítas, la lectura oral y la realización de tareas.
\\

La dislexia \cite{Asandis} constituye el 80\% de los diagnósticos relacionados con trastornos del aprendizaje, teniendo una incidencia entre el 2-8\% de los niños escolarizados. Existe un porcentaje mayor entre los niños y es habitual que los casos cuenten con antecedentes familiares, aún no habiendo sido identificados. Por otra parte, también supone un importante factor de abandono escolar y se sitúa como la más frecuente entre las dificultades en la lectura y aprendizaje. Suele estar asociada al trastorno del cálculo y la expresión escrita y es habitual que las personas afectadas padezcan de problemas de atención e impulsividad.
\\

La identificación de la dislexia, ha provocado debates durante las últimas décadas, dónde se han expuesto numerosas causas para su justificación e incluso se ha llegado a dudar de su existencia.
\\

Los estudios de neuroimagen \cite{LaDislexia} han aportado la creencia de que la dislexia tiene una base neurobiológica. En el cerebro de un dislexico, se produce una alteración durante la formación neuronal, concretamente en el periodo del desarrollo embrionario. Se forman entonces unos cúmulos llamados ectopias, que desorganizan las conexiones implicadas en los procesos de lectoescritura situados en el interior de la corteza. Gracias a la ayuda recibida por los estudios en pacientes fallecidos, se conocen que ciertas áreas del hemisferio izquierdo muestran actividad reducida en las áreas implicadas en el procesamiento de la lectura. Con estos datos, podríamos afirmar entonces que la dislexia tiene un origen neurobiológico, con carga hereditaria y déficit fonológico como causas principales.
\\

Los efectos derivados de la dislexia son múltiples y, dependiendo del individuo, pueden darse en mayor o menor medida y, se pueden observar algunos de los siguientes:
\begin{itemize}
    \item Desinterés por el estudio y caida del rendimiento escolar.
    \item Rechazados por el grupo y llegan a ser considerados con retraso intelectual, pudiendo llegar al caso de verse inmersos en una depresión.
    \item Inadaptación personal, sentimiento de inseguridad y terquedad para someterse a los tratamientos.
    \item Sobreprotección paterna para salvaguardar de los problemas a los niños afectados.
\end{itemize}

Existen diversos tipos de dislexia, en los cuales es importante distiguir entre dislexia adquirida, que aparece a causa de una lesión cerebral, y la dislexia evolutiva dónde el individuo presenta las dificultades asociadas sin una causa concreta que la explique. En ambas se pueden diferenciar otros tipos \cite{WebConsultas} en función de los síntomas predominantes:
\begin{itemize}
    \item \textbf{Fonológica:} dificultad al realizar lecturas globales, deduciendo las palabras conocidas. Se comenten fallos visuales, por ejemplo, entre lobo y lodo, y errores de lexicalización. Esto puede dar lugar a problemas para comprender textos y hace prácticamente imposible leer palabras desconocidas. Un ejemplo visual de esta afección puede verse en \cite{GitDislexia}.
    \item \textbf{Superficial:} dificultad al leer palabras cuya lectura y pronunciación no se corresponden. Afecta de forma más habitual a los niños y tiene una incidencia notable en los paises angloparlantes, debido a que el inglés es un idioma donde las letras no tienen un único sonido.  
    \item \textbf{Profunda o mixta:} en este caso, los dos procesos de lectura, fonológico y visual, se encuentran dañados. Únicamente los casos de dislexia evolutiva se ven afectados. Esto supone una dificultad para descifrar el significado de palabras, además de errores visuales y semánticos.
\end{itemize}


\subsection{Dispraxia}
\subsection{Discalculia}

\section{Crítica al estado del arte}
El mercado tecnológico actual está plagado de sistemas y aplicaciones hechas para simplificar la vida de las personas. Desde realizar una compra o una simple consulta en la web, hasta juegos que permitan desarrollar la capacidad intelectual y creativa de cada individuo.
\\

Por otra parte, el sistema educativo pretende ser inclusivo con una sociedad donde cada uno es diferente y cada vez más. se descubren trastornos y comportaminetos no identificados anteriormente. No siempre se consigue este objetivo, y ante la posibilidad de dejar a alguien atrasado con respecto al resto, se plantean ayudas para evitar que puedan perder el ritmo normal de seguimiento de una clase o una actividad.
\\

Ante el reto que supone ayudar a tantas peronas que sufren de algún síntoma relacionado con la dislexia, se realizan múltiples implementaciones de distintos sistemas, tanto haciendo uso de herramientas tecnológicas como diferentes metodologías docentes. 
\\

En este proyecto nos centraremos en estudiar los sistemas que tratan este trastorno mediante aplicaciones existentes en el mercado u otras propuestas recogidas por diversos órganos de estudio. Gracias a ello, podremos comparar en qué estado se encuentra nuestro objetivo y qué podemos aportar para sumar al beneficio común de esta causa.  

\subsection{Aplicaciones móviles}

\section{Propuesta}
